\documentclass[10pt,a4paper,font=plain]{abnt}
\usepackage[brazil]{babel}
\usepackage[utf8]{inputenc}
\usepackage[alf]{abntcite}


\titulo{Integração Contínua}
\autor{Hugo Lopes Tavares\\
\small Instituto Federal Fluminense\\
\small \texttt{htavares@iff.edu.br}\\
}
\date{04 de março de 2010}

\begin{document}
\maketitle

\begin{resumo}
Com a ascensão de metodologias ágeis, entre elas principalmente Extreme Programming, várias práticas de desenvolvimento de software foram amplamente disseminadas e se provaram eficientes. Dentre tais técnicas estão Desenvolvimento Guiado por Testes \textit{(Test-Driven Development)}  e Integração Contínua, técnicas estas que se completam. Este trabalho abordará alguns conceitos relacionados a integração contínua e apresentará formas de aplicá-la no dia-a-dia de uma equipe, usando uma abordagem automatizada.\\

Palavras-chave: Ágil, Integração Contínua, Sistema de Controle de Versão, Test-Driven Development, Refactoring, Extreme Programming
\end{resumo}



\section{Introdução}

Segundo \citeonline{CI}, o problema de integrar software não é um problema novo, e a medida que a complexidade de um projeto aumenta a necessidade de integrar códigos de diferentes membros de uma equipe cresce. \citeonline{CI} diz que um dos pontos de maior risco no ciclo de desenvolvimento de software é a integração de modificações independentes feitas por diferentes membros de uma equipe.

\citeonline{CI} salienta que esperar até o fim do projeto pra fazer integração leva a todo tipo de problemas de qualidade, que são custosos e na maioria das vezes trazem atrasos para o projeto. Porém, quando a integração é feita continuamente os riscos são abordados mais rapidamente e em pequenos incrementos antes de ir pra produção. Assim, caso ocorra algum problema, a equipe será alertada rapidamente.



\section{Definição de Integração Contínua}

Como o próprio nome já diz, integração contínua significa integrar frequentemente mudanças feitas durante o desenvolvimento de software.
\citeonline{Fowler} define integração contínua da seguinte maneira: "uma prática de desenvolvimento de software onde membros de uma equipe integram seu trabalho frequentemente, normalmente cada pessoa integra pelo menos diariamente" - levando a múltiplas integrações por dia. Cada integração é verificada por um \textit{build} (incluindo teste) para detecar erros de integração o mais rápido possível. \citeonline{Fowler} também relata que muitas equipes acham que essa abordagem leva a significantes quedas de problemas relacionados a integração e permite que uma equipe desenvolva software coeso mais rapidamente.



\section{Pré-requisitos para Fazer Integração Contínua}

\citeonline{Fowler} identifica que integração contínua é simplesmente o processo de integrar código frequentemente - não necessariamente usando um processo automatizado. Porém, como \citeonline{CI} discute, os aspectos automatizados de um processo de integração contínua, que segundo o mesmo trazem muito mais benefícios.
Porém, uma integração contínua não é possível mesmo quando todo o suporte ferramental existe se a equipe não integrar frequentemente código ao repositório de códigos. Assim, antes de mais nada é necessário que a equipe proponha-se a integrar modificações várias vezes ao dia código que funciona, ou seja, código que satisfaz as necessidades e passa em todos os testes.


\subsection{Repositório de Código}

Em qualquer equipe de desenvolvimento é imprescindível o uso de um repositório de códigos e um sistema de controle de versões cuida bem dessa parte - tais como CVS, Subversion, Git, etc - pois além de manter um histórico das alterações do repositório também possibilita baixar versões diferentes dos arquivos.

\subsection{Conjunto Sólido de Testes}

Uma peça fundamental dentro do desenvolvimento ágil de software são práticas de técnicas como Test-Driven Development \cite{TDD} e Behaviour-Driven Development \cite{BDD}. Usando tais práticas é possível criar um conjunto sólido de conjuntos de testes que tornem possível validar o estado do software frequentemente. É possível identificar problemas de forma automatizada, trazendo benefícios futuros ao projeto, tais como redução no custo da validação do software.


\subsection{Build Automatizado}

Um \textit{script de build} automatizado desempenha um papel importantíssimo no ciclo de vida de um projeto, pois com ele é possível que todos consigam instalar o projeto e rodar todos os testes em diferentes máquinas. Usar um script de build significa ter um \textit{script} que cuida de baixar e instalar dependências, compilar códigos-fontes e é responsável por rodar os testes e até mesmo rodar ferramentas de cobertura de teste, notificando caso ocorra erros ou falhas no procedimento.


\subsection{O Ambiente de Produção Deve Ser Clonado}

Como descrito por \citeonline{Fowler} os testes devem ser executados numa máquina que possua um ambiente idêntico ao do ambiente de produção. O ponto principal é que usando o mesmo ambiente é possível evitar problemas antes do software ir pra produção. Tanto problemas relacionados a versões de dependências quanto a sistemas operacionais. Essa prática se mostra eficiente pois a equipe não tem surpresas na hora de implantar o sistema no ambiente de produção, pois o ambiente é idêntico ao ambiente em que o software foi desenvolvido e testado.


\subsection{Resumo das Práticas}

Usando todas as práticas citadas anteriormente é possível: compartilhar e versionar código entre a equipe, validar o sistema a qualquer momento, instalar e rodar os testes facilmente e evitar surpresas na hora da implantação no ambiente de produção. Vale lembrar que essas práticas são as mais fundamentais quando se fala em integração contínua. Todas devem andar juntas e são um pré-requisito pra qualquer equipe que queira usar integração contínua.



\section{Integração Contínua Síncrona, Assíncrona}

Segundo \citeonline{XP} há dois modelos de integração contínua: síncrono e assíncrono.
No modelo síncrono o \textit{build} é executado na máquina de integração pela equipe assim que um conjunto de alterações está pronto pra ir pro repositório de código. As alterações são baixadas pra máquina, o build é executado e caso seja executado com sucesso, passando em todos os testes e com cobertura de testes satisfatória (caso cobertura seja importante pra equipe), enviado ao repositório de código.
O modelo assíncrono é mais comum, segundo \citeonline{XP}. Nesse modelo os conjuntos de alterações feitas pela equipe são enviados ao repositório de código e uma máquina - chamada máquina de integração - é responsável por executar o \textit{build} sempre que novas alterações forem identificadas no repositório. Esta mesma máquina é responsável por notificar a equipe se ocorreu algum problema como erros, falhas ou cobertura insatisfatória dos testes - seja por e-mail, mensagem de texto, IRC ou qualquer outro meio de comunicação. \citeonline{Fowler} usa os termos build manual e servidor de integração referindo-se respectivamente a integração síncrona e assíncrona.

\citeonline{Teles} diz que um problema no modelo síncrono é que ele funciona melhor quando a equipe inteira está num mesmo local, pois estabelece-se um acordo sobre a vez de integrar e evita-se que o repositório de código fique com código inconsistente em qualquer momento.
O modelo assíncrono, como descrito por \citeonline{Teles}, é recomendado pra projetos onde a equipe está distribuída, ou seja, não estão todos no mesmo local físico. \citeonline{Teles} identifica esta abordagem como ideal para projetos de software livre e open source, projetos estes que na maioria das vezes a equipe está espalhada geograficamente.
Comparando os dois modelos de integração contínua \citeonline{Teles} diz usar o modelo síncrono na empresa ImproveIt e diz que o modelo assíncrono é mais arriscado, pois há o risco de em algum determinado momento deixar o repositório de códigos com inconsistências durante algum tempo, pois a equipe nem sempre resolve em tempo hábil o problema, mesmo sendo alertada rapidamente, e pode ser que nesse meio tempo alguém use código inconsistente.
Em ambos modelos podem haver uma máquina real ou virtual dedicada a integração das alterações.

No modelo de integração contínua de \citeonline{CI} há sempre um servidor de integração contínua responsável por observar as alterações no repositório de códigos, e sempre que acontecer alguma alteração o servidor executará o build e notificará a equipe caso haja inconsistências, da mesma maneira como descrita no processo assíncrono de \citeonline{XP}. \citeonline{CI} deixa claro que o uso de uma máquina dedicada a integração de software é fundamental para um bom processo de integração contínua, não falando hora nenhuma sobre mais de um modelo de integração como os descritos por \citeonline{XP} ou \citeonline{Fowler}.

Uma das vantagens de possuir servidores de integração contínua que monitoram o repositório de código é poder ter builds executados em diferentes sistemas operacionais e até mesmo arquiteturas, evitando, assim, o problema do desenvolvedor dizer "mas na minha máquina funciona", como descrito em \citeonline{Kniberg}.

\section{Ferramentas}

Atualmente há várias ferramentas disponíveis para esta tarefa. Entre as mais famosas estão CruiseControl.rb\footnote{http://cruisecontrolrb.thoughtworks.com/ acesso em 04 de março de 2010}, Hudson\footnote{http://hudson-ci.org/ acesso em 04 de março de 2010} e BuildBot\footnote{http://buildbot.net/trac acesso em 04 de março de 2010} - todas ferramentas open-source. Porém, ferramentas mais simples, tais como Make\footnote{http://www.gnu.org/software/make/ acesso em 04 de março de 2010}, Rake\footnote{ http://rake.rubyforge.org/ acesso em 04 de março de 2010}, Ant\footnote{http://ant.apache.org/ acesso em 04 de março de 2010} e Maven\footnote{http://maven.apache.org/ acesso em 04 de março de 2010} e Buildout\footnote{http://www.buildout.org acesso em 04 de março de 2010} podem ser usadas pra criar builds de forma simples. A diferença das últimas ferramentas em comparação com as primeiras é que essas são mais completas, podendo até mesmo cuidar de envio de e-mails, agendar builds e até comandar diferentes máquinas para executarem builds.

\section{Conclusão}

A prática de usar integração contínua se mostra muito eficaz nas equipes de desenvolvimento e está sendo usada amplamente. Inúmeros projetos open-source usam várias máquinas dedicadas a serem servidores de integração, rodando em muitas vezes sistemas operacionais diferentes ou diferentes versões de softwares, disponibilizando sempre um relatório para a equipe. Em equipes onde há distância geográfica recomenda-se o uso de integração contínua assíncrona e em equipes que trabalham no mesmo espaço físico recomenda-se o uso de integração contínua síncrona. Em ambos casos é importante ter pelo menos uma máquina dedicada a integração que seja um clone do ambiente de produção, pois o quanto mais rápido for o \textit{feedback} em um projeto, melhor.

Em suma, integração contínua resume-se em ter \textit{feedback} rápido e assegura que sempre existirá um software pronto pra ser colocado em produção (obviamente não está com todos os recursos, mas deverá sempre ser usável). \citeonline{XP} define \textit{feedback} como um dos valores mais importantes no método \textit{Extreme Programming} \cite{XP}.
\bibliographystyle{abnt-alf}
\bibliography{referencias}

\end{document}
