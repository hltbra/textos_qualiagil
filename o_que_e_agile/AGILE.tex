% Notas de aprendizado: 
% ldots{} é => ...
% Em tipografia, use dois ` (acento grave) para abrir aspas e dois ' (apóstrofe) para fechar. 
% CITAÇÃO: \citeonline e \cite

\documentclass[a4paper,11pt,abntfigtabnum,noindentfirst]{abnt}
\usepackage[utf8x]{inputenc} %acentuaçao
\usepackage[alf]{abntcite} 

\begin{document}
\chapter{O que é \textsl{agile}?}

\section {Introdução}

Por se tratar de um processo complexo, que envolve criatividade e muito influenciado também pela cultura de cada equipe, o desenvolvimento de software acaba estando fadado a enfrentar diversas dificuldades, principalmente relacionadas a estipulação de custos e prazos. Assim, para o sucesso de um projeto de software, a determinação da metodologia e, por conseguinte, do método mais adequados a equipe é um ponto essencial.

A mais recente pesquisa realizada pelo Standish Group \citeonline{Chaos2009}, de 2009, mostra que somente cerca de 32\% dos projetos respeitaram prazos, custos e todas as funcionalidades especificadas em sua entrega, cerca de 24\% foram cancelados antes de seu fim e 44\% necessitaram de prazos maiores, custos maiores ou menos funcionalidades do que especificado no início do projeto para serem entregues. Confrontadas com essa realidade, as equipes vem buscando (ou pelo menos, deveriam estar buscando) novas formas de gerenciamento e organização para trazer maior sucesso aos projetos.

Uma vertente que surgiu nesse movimento de busca de mudanças foi a metodologia ágil (conhecida pelo termo em inglês \textsl{agile}) que, assim como a metodologia tradicional, têm como objetivo reger da forma mais eficiente projetos de desenvolvimento de software, porém os métodos existentes sugerem práticas muito diferentes para atingir esse fim.

Nas duas últimas décadas, o mercado de software vem aceitando, entendendo e aplicando a metodologia ágil como uma alternativa \citeonline{XPExplained}, essa aceitação tem sido apoiada e incentivada pelo avanço das tecnologias, uma vez que a maioria das linguagens utilizadas atualmente permitem que se apliquem novas técnicas de comprovada eficiência para garantir a qualidade de software. Dentre as técnicas essenciais pregadas pelo movimento ágil, estão: iterações curtas, testes automatizados e integração contínua.

O perfil e a cultura da equipe devem ser avaliados ao definir se um método ágil é mais ou menos adequado para ela. Também o tamanho da equipe deve ser levado em conta. Para decidir com mais clareza o que mais se adapta a equipe e o que trará maior qualidade e eficiência ao processo e consequentemente ao produto final, é importante conhecer bem as vertentes existentes. Esse texto vem introduzir conceitos importantes de ágil, suas práticas e príncipios e também sugerir práticas e ferramentas para tornar possível a implementação nas equipes, de forma prática. Ele faz parte de um conjunto de materiais, dentre eles textos, exemplos de código e screencasts com esse mesmo objetivo.

\section {O que é \textsl{agile}?}

As diretrizes e princípios da metodologia ágil estão descritos num documento, que foi escrito em fevereiro de 2001 por um grupo de especialistas da área, conhecido como manifesto ágil \citeonline{AgileManifesto}. A partir desse documento e seguindo suas indicações, foram criados os métodos ágeis de desenvolvimento de software, que tem valores e princípios comuns, apesar de utilizarem práticas diferentes.

Esse documento estabelece 4 valores fundamentais que redefinem a maneira como deve ser entendido o processo de construção de software:

	* Indivíduos e interações devem estar acima de processos e ferramentas. 

	* Software funcionando é um resultado mais relevante do que documentação completa e detalhada.

	* Participação do cliente e desenvolvimento colaborativo é mais importante do que negociação de contratos.

	* O projeto estar apto a se adptar a mudanças é mais importante do que seguir seguir o plano inicial.

No mesmo documento, foram definidos alguns princípios da metodologia ágil:

   1. "Nossa maior prioridade é satisfazer o cliente através da entrega rápida e contínua de software de valor.''
    
   2. ``Entregue frequentemente software que funciona, de algumas semanas a alguns meses, com preferência para um período curto.''
   
   3. ``Software que funciona é a medida primária de progresso.''

   4. ``Pessoas de negócio e os desenvolvedores devem trabalhar juntos diariamente durante o projeto.''

   5. ``A maneira mais eficiente e efetiva de transmitir informação para uma equipe de desenvolvimento é via conversas cara-a-cara.''

   6. ``Em intervalos regulares, a equipe reflete sobre como se tornar mais efetiva, então refina e ajusta seu comportamento de acordo.''

   7. ``Receba mudanças de requerimento, mesmo tarde no desenvolvimento. Processos Ágeis esperam mudança que tragam mais vantagem competitiva ao cliente.''

   8. ``Construa projetos ao redor de pessoas motivadas. Lhes dê o ambiente e suporte que precisam, e confie neles para realizar o trabalho.''

   9. ``Processos Ágeis promovem desenvolvimento sustentado. Os patrocinadores, desenvolvedores e usuários devem ser capazes de manter um ritmo constante indefinidamente.''

  10. ``Atenção constante à excelência técnica e bom design aumenta a agilidade.''

  11. ``Simplicidade – a arte de miximizar a quantidade de trabalho não feito – é essencial.''

  12. ``As melhores arquiteturas, requerimentos e designs emergem de equipes auto-organizadas.''

É um erro tentar executar num projeto apenas um ou alguns desses valores e princípios, quando na verdade, a junção das idéias trazidas pelos mesmos é que vai propiciar o sucesso. Optar por utilizá-los em separado pode acabar se tornando uma ponte para o fracasso. \citeonline{PrincipiosAgeis}

\begin{center}\textsl{``as práticas técnicas trazidas pelos métodos ágeis - iterações curtas, desenvolver testes antes de código, integração contínua - não são opcionais. Assuma os riscos de ignorá-las ou deixá-las para mais tarde.''} \citeonline{BecomingAgile} \end{center}

A junção dos príncipios e valores ágeis levaram ao desenvolvimento dos métodos ágeis, que são um conjunto de técnicas para o desenvolvimento de software, criadas pelos mesmos autores do manifesto que originou a metodologia. Assim,  os métodos se diferenciam nas práticas, mas partilham dos mesmos valores e princípios. São elas: XP (eXtreme Programming), DSDM (Dynamic Systems Development Method), Família Crystal, ASD (Adaptive Software Development), SCRUM, FDD (Feature-driven development).

As práticas essenciais dos métodos ágeis podem ser definidas em:

    * Comunicação direta com o cliente inicialmente, para definição de requisitos e planejamento das iterações. O cliente faz a priorização de requisitos e a equipe tenta atender e desenvolver sempre primeiramente o que tem mais valor para ele. 

    * Contratos de escopo negociável \citeonline{EscopoNegociavel}, prazos e metas definidos em prazos curtos, para que não haja 'chutes' e demora exagerada na entrega, além de permitir que o cliente verifique os resultados e possa decidir sobre a continuação ou não do projeto após a entrega de cada release a prazos combinado. Objetiva também aumentar o comprometimento da equipe com o resultado esperado.

    * Cada iteração é planejada há prazos curtos, de forma que ao fim haja um conjunto de material considerado 'pronto'. O planejamento é feito respeitando o valor e a complexidade do que foi requerido pelo cliente.

    * Documentação deve ser utilizada sempre que agregar valor ao produto. Porém as equipes procuram automatizar o máximo possível, pois entendem que isso facilita a manutenibilidade do sistema e permitem ao projeto se adequar mais facilmente a mudanças, que são entendidas como parte do processo.

    * Testes devem estar em primeiro lugar(test-first), e não deixados para o fim do ciclo como é no processo tradicional. O objetivo é detectar erros no início, facilitando sua correção e diminuindo o impacto das mudanças. As mudanças se tornam parte do processo, uma vez que deve ser feito sempre o mínimo necessário para completar uma iteração, para que ao fim da mesma, o conceito de 'pronto' seja atingido.

    * Os testes devem ser automatizados, para permitir test-first. Essa prática possibilita rastreabilidade de requisito, diminuição de impactos, verificação da cobertura de testes no código e é uma prática necessária para atingir a integração contínua. REFERENCIAPARACAPITULOINTEGRACAOCONTINUA

    * Integração da equipe, com reuniões em intervalos regulares para ajustar o comportamento da equipe. A comunicação é direta é de extrema importância, não só com o cliente, mas entre toda a equipe. Para tanto, algumas vezes são aplicadas técnicas de reuniões 'relâmpago' diárias, para verificação de avanços, programação em par para divisão de experiências, equipe onde os papéis são invertidos de acordo com o conhecimento individual sobre o problema... Assim, cooperação, comprometimento e colaboração são exigidos da equipe. 

    * O desenvolvimento deve ser continuamente sustentado, onde equipe, cliente e usuários possam estabelecer um ritmo de crescimento constante. Esse crescimento é auto-organizado, e é possibilitado pela visualização de que a colaboração é necessária para o sucesso individual, trazida pelo processo como um todo.

Além de práticas técnicas, existe a necessidade de mudança na maneira de pensar sobre desenvolvimento e organização das equipes envolvida na adoção de algum dos métodos da metodologia ágil. As vantagens macro são que o projeto se torna mais adaptável a mudança, as pessoas ficam mais integradas, os prazos são respeitados com mais rigidez, produto de valor para o cliente é entregue com frequência, a qualidade do código em si tende a aumentar, pois são considerados valores de design e devido aos testes automatizados.

Existem algumas variantes importantes entre os métodos ágeis, e elas podem ser vistas em detalhes em livros como (Extreme Programming Explained, Scrum e XP direto das trincheiras, algum livro de Lean(ver com Fernando, Rodrigo....))

\section {O que não é \textsl{agile}}

\begin{center}\textbf{ágil} adj \textsl{(lat agile)} 1 Desembaraçado, destro, ligeiro, vivo. 2 Flexível, leve. \citeonline{Dicionario} \end{center}

O mais comum entre pessoas que conhecem metodologia ágil superficialmente, é imaginar que ao utilizar um método ágil descarta-se totalmente documentação, planejamento, estipulação cuidadosa de contratos, e portanto, que o processo é bagunçado, as pessoas decidem executar tarefas não planejadas, para ser rápido e não se preocupam em guardar registros. Porém, isso não é usar metodologia ágil. Isso é usar nenhuma metodologia. 

A palavra ágil, por definição, sugere um processo de desenvolvimento leve e flexível. Isso não significa desordem. Significa que durante todo o processo de desenvolvimento a equipe estará preocupada em desenvolver o que gera mais valor para o cliente, ter um conjunto de material pronto a cada iteração, utilizar um processo mais colaborativo e menos burocrático, tornando-o mais mais leve e tendo mais asserção no que está sendo desenvolvido.

Contratos, documentação e planejamento são parte do processo de desenvolvimento ágil, porém acima disso, estão as interações entre indivíduos, e software que funciona sendo desenvolvido. Isso é mostrado claramente no manifesto ágil, pelo constraste entre os itens da direita e da esquerda da lista dos 4 valores que devem estar presentes ao longo do desenvolvimento.

A experiência das equipes de desenvolvimento e dos gerentes de projetos de software (colocar links para experiencias aqui, xp das trincheiras, alguma do vinicius teles, alguma de outros caras legais aí, do getting real) relatam que escolher trabalhar com um metódo ágil é escolher utilizar a prática de testes automatizados para desenvolver software mais seguro, rápido e manutenível, guardar o próprio software para utilizá-lo como rastro de mudanças, através do armazenamento de suas versões, utilizá-lo como ferramenta para rastreabilidade de requisitos, automatizando e aumentando a qualidade do próprio software. 

O desafio não é o de tornar o processo rápido e simples, mas sim, procurar colocar a complexidade no lugar certo. E a maior complexidade, a maior diferença comparando a produção de bens tangíveis, o valor final e o grande objetivo é o software.

\bibliographystyle{abnt-alf}
\bibliography{mybib}

\end{document}

\begin{comment}
PRIMEIRO DE TUDO, DEFINI ASSIM:

Metodologia (Estudo científico dos métodos) ágil  => a metodologia: manifesto e princípios
Ágil X Tradicional

Métodos (conjunto de meios convenientes para alcançar um fim específico, completo, -sozinho, resolve-) Ágeis
XP, Scrum, Lean, FDD

Em tradicional: RUP, Prototipação . . .

Técnicas (práticas) ágeis (tem de estar inseridas num método... sozinhas, são práticas):
BDD, TDD, INTEGRAÇÃO CONTÍNUA, PAR PROGRAMMING

Em tradicional: Documentação, pontos por caso de uso, pontos por função, as paradas de gerencia de configuração (...)

Objetivo... Anterior, para linkar os nossos textos... So comecei a escrever!

Serão apresentadas práticas que visam tornar o software mais confiável e manutenível, seguindo os preceitos ágeis, com ênfase para o desenvolvimento orientado a comportamento (BDD) e outras alternativas agéis à metodologia tradicional.

Nesse trabalho, o objetivo é demonstrar como é possível aumentar a qualidade do software produzido através do uso de uma técnica específica, o Behaviour Driven Development (BDD).......... (acho que vou ler o dos meninos primeiro ao todo para depois completar isso aqui, que vai ser tipo a introdução....)

um texto que li e achei legal o debate (comentários):
http://imasters.uol.com.br/artigo/14565/desenvolvimento/seja_inteligente_e_nao_use_agile/

PPT sobre as métodos ágeis:

http://www.google.com.br/url?sa=t&source=web&ct=res&cd=1&ved=0CAYQFjAA&url=http%3A%2F%2Fxp.edugraf.ufsc.br%2Fpub%2FXP%2FOutrosProcessosAgeis%2FMetodologiasgeis-Sabrina.ppt&ei=5d2HS_qmDYGRuAfnla2DDA&usg=AFQjCNGThz51tadFwdiOLxtm0F9SWbHxxQ&sig2=3qjbqfxNBxaohlULiVnZrQ

=====================================================

MPS.BR:

FALAR DOS PODCASTS COM GUSTAVO.... SE ELE QUISER, ELE TINHA DITO Q QUERIA MATERIAL DO MPS E EU ACABEI ESQUECENDO!

A PARADA QUE A GENTE FOI VENDO É QUE

   1.      INSTITUIÇÕES IMPLEMENTADORAS NEM SEMPRE ENTENDEM A FLEXIBILIDADE QUE MPS E CMMI IMPOEM QUANDO DIZEM O QUE E NÃO COMO FAZER, E ACABAM COBRANDO RESPOSTAS INADEQUADAS AO PERFIL DA EQUIPE TORNANDO O PROCESSO PENOSO

   2.      EQUIPES DE DESENVOLVIMENTO AGILE TEM POR NATUREZA UMA REJEIÇÃO A METODOLOGIA TRADICIONAL E MUITAS VEZES, PECAM EM EVITAR ALÉM DA CONTA FORMALIZAR O PROCESSO (E MPS E CMMI TEM CARA DE FORMALISMO, PAPEL... PELA MANEIRA COMO SÃO MUITAS VEZES DISCUTIDOS E IMPLEMENTADOS NAS EQUIPES (ENGESSAR O PROCESSO E TAL)

AGILE: 
Falar da idéia do evento palestras agile + jogos agile + dojo

O desafio não é tentar tornar o processo rápido e simples, o desafio é colocar a complexidade no lugar certo.

\end{comment}